\documentclass{book}
\usepackage{listings,color,textcomp}
\usepackage{url}

\begin{document}

\lstset{
  language=Scheme,
  morekeywords={extern, kernel, reduce, vector, void, int, u64, float, str}
}

\title{Harlan User's Guide}

\maketitle

\tableofcontents

\chapter{Getting Started}

Harlan is known to build and run on the following operating systems.
\begin{itemize}
\item Mac OS X 10.6 (Snow Leopard)
\item Mac OS X 10.7 (Lion)
\item Gentoo Linux
\end{itemize}

Others will probably work as well.

Harlan requires an OpenCL implementation as well as Petite Chez
Scheme. Below are several OpenCL implementations that should work.

\begin{itemize}
\item Intel OpenCL SDK
  \footnote{\url{http://software.intel.com/en-us/articles/opencl-sdk/}}
\item NVIDIA CUDA Toolkit
  \footnote{\url{http://developer.nvidia.com/cuda-toolkit}}
\item AMD Accelerated Parallel Processing (APP) SDK
  \footnote{\url{http://developer.amd.com/SDKS/AMDAPPSDK/Pages/default.aspx}}
\end{itemize}

Petite Cheze Scheme can be downloaded from
\url{http://www.scheme.com}.

Once all the prerequisites are installed, you can now build
Harlan. The source code for Harlan is hosted on Indiana University's
private GitHub. The following commands will check out the source code,
build Harlan, and run the test suite.

\begin{verbatim}
    git clone git://github.iu.edu/eholk/harlan.git
    cd harlan
    make check
\end{verbatim}

If the tests are successful, you will see the following at the end of
all the output.

\begin{verbatim}
    All tests succeeded.
\end{verbatim}

The test programs are available in the \texttt{test}
directory. End-to-end test programs have the \texttt{.kfc}
extension. Other extensions represent code that is valid at various
intermediate passes in the compiler.

Make puts test binaries in the \texttt{test.bin} directory, and also
saves output from test programs here. Programs may be run directly
from this directory for easier debugging.

Harlan programs can be compiled manually as follows.

\begin{verbatim}
    ./harlanc hello.kfc
\end{verbatim}

For debugging purposes, the \texttt{-v} flag can be used.

\begin{verbatim}
    ./harlanc -v hello.kfc
\end{verbatim}

This causes the compiler to write out the intermediate results from
each compiler pass.

Assuming the Harlan compiler is successful, the compiler will dump C++
code to the terminal. This can be saved in a file or piped directly
into a C++ compiler to produce a running executable.

If Latex is installed, you can build the Harlan User's Guide (which
you are currently reading) as follows.

\begin{verbatim}
    make docs
\end{verbatim}


\chapter{Using the Harlan Compiler}

\chapter{The Harlan Language}

\chapter{The Runtime Library}

Harlan provides a small library of built-in functions. These are
described in this chapter. Each of these may be used in a program by
adding the appropriate \lstinline{extern} definition.

\section{nanotime}

\textbf{Prototype:} \lstinline{(extern nanotime () -> u64)}

This function returns the number of nanoseconds that has elapsed since
January 1, 1970. It is useful for timing benchmark programs.

\section{print\_2x2\_int\_vec}

\textbf{Prototype:} \lstinline{(extern print_2x2_int_vec (vector (vector int 2) 2) -> void)}

Formats a $2 \times 2$ matrix and prints it to standard out.

% TODO: include example output

\section{write\_pgm}

\textbf{Prototype:} \lstinline{(extern write_pgm str (vector (vector int 1024) 1024) -> void)}

This function treats its second argument as a $1024 \times 1024$
grayscale image and writes it to a Portable GrayMap (PGM) file named
by the function's string argument. Each entry in the image should be
between 0 and 255, inclusive. The resulting file may then be viewed by
a program such as ToyViewer on the Mac.

This function is used by the Mandelbrot set test cases to allow us to
visually confirm the program is behaving correctly.

\end{document}
