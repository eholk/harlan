\documentclass{book}
\usepackage{listings,color,textcomp}

\begin{document}

\lstset{
  language=Scheme,
  morekeywords={extern, kernel, reduce, vector, void, int, u64, float, str}
}

\title{Harlan User's Guide}

\maketitle

\tableofcontents

\chapter{Getting Started}

\chapter{Using the Harlan Compiler}

\chapter{The Harlan Language}

\chapter{The Runtime Library}

Harlan provides a small library of built-in functions. These are
described in this chapter. Each of these may be used in a program by
adding the appropriate \lstinline{extern} definition.

\section{nanotime}

\textbf{Prototype:} \lstinline{(extern nanotime () -> u64)}

This function returns the number of nanoseconds that has elapsed since
January 1, 1970. It is useful for timing benchmark programs.

\section{print\_2x2\_int\_vec}

\textbf{Prototype:} \lstinline{(extern print_2x2_int_vec (vector (vector int 2) 2) -> void)}

Formats a $2 \times 2$ matrix and prints it to standard out.

% TODO: include example output

\section{write\_pgm}

\textbf{Prototype:} \lstinline{(extern write_pgm str (vector (vector int 1024) 1024) -> void)}

This function treats its second argument as a $1024 \times 1024$
grayscale image and writes it to a Portable GrayMap (PGM) file named
by the function's string argument. The resulting file may then be
viewed by a program such as ToyViewer on the Mac.

This function is used by the Mandelbrot set test cases to allow us to
visually confirm the program is behaving correctly.

\end{document}
